\PassOptionsToPackage{english}{translator}
\documentclass{beamer} %[handout, notes=only]

\usepackage{etex} % Porque sino al usar muchos paquetes da error.

\usepackage[english]{babel}

\mode<presentation> {
\usetheme{Madrid}
}

\usepackage[utf8]{inputenc}
\usepackage{amsmath}
\usepackage{graphicx}
\usepackage{booktabs} % Allows the use of \toprule, \midrule and \bottomrule in tables
\usepackage{multicol}

\usepackage[export]{adjustbox} % Para bordes en las imágenes.

\usepackage{svg}

\usepackage{amsfonts}

\usepackage{array}

\setkeys{Gin}{height=7cm}

\graphicspath{{img/}}

\DeclareMathOperator*{\argmax}{arg\,max}

\AtBeginSection{
    \begin{frame}
        \frametitle{Agenda}

        \begin{multicols}{2}
            \tableofcontents[currentsection, hideallsubsections]
        \end{multicols}
    \end{frame}
}

\title[Is This a Joke?]{Is This a Joke? Humor Detection in Spanish Tweets}

\author[Castro, Cubero, Garat and Moncecchi]{Santiago \textsc{Castro}, Matías \textsc{Cubero}, Diego \textsc{Garat} and Guillermo \textsc{Moncecchi}}
\institute[]{
    NLP Group, University of the Republic, Uruguay
}
\date{November 24, 2016}

\usepackage{pgfpages}

\mode<handout>{
  \pgfpagesuselayout{4 on 1}[a4paper,landscape,border shrink=2.5mm]
}

\begin{document}

\begin{frame}
    \titlepage{}
\end{frame}

\note{Bienvenidos a la presentación de nuestro proyecto de grado. Mi nombre es Santiago y él es Matías. El tema a tratar es Detección de humor en textos en español, es decir, lo que se hace en este trabajo es reconocer humor en textos en idioma español.}

\begin{frame}
    \frametitle{Agenda}

    \begin{multicols}{2}
        \tableofcontents[hideallsubsections]
    \end{multicols}
\end{frame}

\note{Este es el cronograma. Primero vamos a ver una introducción, incluyendo nuestra motivación y objetivos. Luego vamos a ver qué es humor según varias teorías y cuál es el Estado del arte, es decir, cuál es el estado de resolución de esta tarea en el mundo desde el punto de vista computacional. Vamos a ver cómo recabamos el corpus, el conjunto de datos de texto para poder trabajar. Siguiendo, vamos a tener un poco de contexto sobre el área que se aplica a esta tarea, Aprendizaje Automático. Luego cómo es nuestra implementación y finalmente las conclusiones.}

\section{Introducción} 

\subsection{Motivación}

\begin{frame}[allowframebreaks]
    \frametitle{Motivación}

    \begin{itemize}
        \item La risa caracteriza al ser humano como especie.
        \item Componente esencial en la comunicación humana.
        \item Humor como pieza fundamental en la interacción persona-computadora.
    \end{itemize}

    \framebreak
    
    \begin{itemize}
        \item El humor ha sido estudiado desde el punto de vista psicológico, cognitivo y lingüístico.
        \item ¿Pero desde el punto de vista computaciónal?
        \item Paso intermedio para la generación.
    \end{itemize}
    
\end{frame}
\note{Algunos trabajos previos existen, pero se está aún lejos de concretar una caracterización del humor que permita su reconocimiento y generación automática.}

\subsection{Humor}

\begin{frame}
    \frametitle{Humor}

    \begin{itemize}[<+->]
        \item ¿Qué es el humor?
        \item No es algo objetivo ni fácil de definir
        \item Que lo defina la gente en base a votaciones
    \end{itemize}
\end{frame}

\subsection{Tipo de textos}

\begin{frame}
    \frametitle{Tipo de textos}

    \begin{itemize}[<+->]
        \item En un texto largo hay que determinar el alcance
        \item Los \emph{tweets} son cortos y fácilmente conseguidos

        \begin{itemize}
            \item Usualmente con errores gramaticales
        \end{itemize}
    \end{itemize}
\end{frame}

\subsection{Dificultad}

\begin{frame}
    \frametitle{Dificultad}
    
    --- ¿Tenés WiFi? \\
    --- Claro. \\
    --- ¿Cuál es la clave? \\
    --- Tener dinero y pagarlo. \\
\end{frame}

\subsection{Objetivos}

\begin{frame}
    \frametitle{Objetivos}
    \begin{itemize}
        \item Construir un clasificador de humor en textos en español utilizando métodos de aprendizaje automático
            \begin{itemize}
                \item En particular en tweets
            \end{itemize}
        \item Construir un corpus de tweets en español
    \end{itemize}
\end{frame}

\section{Estado del arte} % A subsection can be created just before a set of slides with a common theme to further break down your presentation into chunks

\begin{frame}
\frametitle{Paragraphs of Text}
Estado del arte
\end{frame}

\subsection{Humor - Teorías}
\begin{frame}
\frametitle{Paragraphs of Text}
Humor - Teorías
\end{frame}

\subsection{Enfoques computacionales}
\begin{frame}
\frametitle{Paragraphs of Text}
Enfoques computacionales
\end{frame}

\section{Construcción del corpus}

\subsection{Extracción}
\begin{frame}
    \frametitle{Extracción}

    \begin{itemize}
        \item Humorístico: se busca en Twitter por la palabra clave \emph{chistes} y se eligen cuentas, llegando a 16.488 tweets.
        \item No humorístico: cuentas de noticias, frases filosóficas y curiosidades, alcanzando los 22.875 tweets.
    \end{itemize}
\end{frame}

\subsection{Anotación}
\begin{frame}
    \frametitle{Anotación}

    \begin{itemize}
        \item Naturalmente se etiquetarían los tweets según su tipo de cuenta; pero se encuentran muchas inconsistencias.

        \begin{itemize}
            \item Ejemplo: \emph{RT @MichelPesquera: El significado de ``Nada'' http://t.co/L1e2uHOhM4}

            \item Hay que anotarlos a mano.
        \end{itemize}

        \item Excesiva cantidad de tweets para anotar.

        \begin{itemize}
            \item Se crea una aplicación para que usuarios los anoten.
        \end{itemize}
    \end{itemize}

    \vspace{1cm}

    \begin{center}
        \bf
        Los usuarios definen al humor.
    \end{center}
\end{frame}

\begin{frame}
\frametitle{Anotación}
\framesubtitle{Cantidad de clases a considerar}
        \begin{center}
        \begin{columns}[c]
            \begin{column}[c]{0.50\textwidth}
                \centering
                \includegraphics[frame, height=1.8cm, width=5.5cm]{pagina-dos-opciones.png}
            \end{column}
            
            \begin{column}[c]{0.50\textwidth}
                \centering
                \includegraphics[frame, height=3.5cm, width=5.5cm]{pagina.png}
            \end{column}
        \end{columns}
    \end{center}
\end{frame}
\begin{frame}
    \frametitle{Anotación}
    \framesubtitle{Otras consideraciones}

    \begin{itemize}
        \item Contenido explícito

        \begin{itemize}
            \item Se busca la mayor cantidad de votantes
            \item Se elimina contenido explícito
        \end{itemize}

        \item Eficiencia

        \begin{itemize}
            \item En el servidor
            \item En tiempo entre tweets
        \end{itemize}

        \item Algoritmo de selección

        \begin{itemize}
            \item Aleatorio
            \item Sin repetir
        \end{itemize}
    \end{itemize}
\end{frame}
\begin{frame}
\frametitle{Anotación}
\framesubtitle{Aplicación Final}
    \begin{center}
        \begin{columns}[c]
            \begin{column}[c]{0.45\textwidth}
                \centering
                \includegraphics[frame, height=3.5cm]{pagina.png}
            \end{column}

            \begin{column}[c]{0.45\textwidth}
                \centering
                \includegraphics[frame, height=7cm]{app.png}
            \end{column}
        \end{columns}
    \end{center}
\end{frame}

\subsubsection{Resultado de la anotación}
\begin{frame}[allowframebreaks]
    \frametitle{Resultado de la anotación}

    \begin{itemize}
        \item[+] 60k votaciones recibidas
        \item[--] 20k votaciones eliminadas
        \item[--] 6,5k votaciones “ver otro” (\emph{“skip”})
        \item[=] 33,5k votos considerados
    \end{itemize}

    \framebreak

    \begin{center}
        \includegraphics{votos_por_calificacion_torta.png}

        \includegraphics{histograma.png}
    \end{center}
\end{frame}

\subsubsection{Humor según la votación}

\begin{frame}[allowframebreaks]
    \frametitle{Humor según la votación}

    Porcentaje de votos positivos: $\frac{\#\{votos\ positivos\}}{\#\{total\ votos\}}$

    \begin{center}
        \includegraphics[height=3.75cm]{histograma_porcentaje_humor.png}
    \end{center}

    \begin{itemize}
        \item $[60\%, 100\%] \Rightarrow$ \textbf{Humor}
        \item $(30\%, 60\%) \Rightarrow$ \textbf{Dudoso}
        \item $[0\%, 30\%] \Rightarrow$ \textbf{No humor}
    \end{itemize}

    \framebreak

    \begin{center}
        \includegraphics[height=6.5cm]{grupos.png}
    \end{center}
\end{frame}

\subsubsection{Concordancia entre los anotadores}

\begin{frame}[allowframebreaks]
    \frametitle{Concordancia entre los anotadores}

    \begin{itemize}
        \item Se quiere saber qué tan de acuerdo estuvieron las personas a la hora de votar.
        \item Se utiliza la medida kappa de Fleiss.
        \item kappa evalúa cuán mejor es la votación respecto a una al azar, siendo lo mejor posible 1 y siendo 0 una votación al azar.
    \end{itemize}

    \note{Interesa saber cuál fue la concordancia entre los anotadores, es decir, qué tan de acuerdo estuvo la gente a la hora de anotar como humor (1, 2, 3, 4 o 5 estrellas) o como humorístico a los tweets. Se propone utilizar la medida kappa. Esta medida se fija la cantidad de pares de anotadores que están de acuerdo en cada categoría para cada tweet. El resultado es un número, en donde 0 significa una anotación hecha al azar y 1 es el mejor acuerdo posible.}

    \framebreak

    \begin{center}
        \begin{tabular}{ c | r | c }
            tweets considerados & \#tweets & $\kappa$ \\
            \hline
            $\geq$2 votos & 8.320 & 0,612 \\
            $\geq$3 votos & 4.309 & 0,523 \\
            $\geq$4 votos & 2.273 & 0,469 \\
            $\geq$5 votos & 1.331 & 0,434 \\
            $\geq$6 votos & 805 & 0,406 \\
            $\geq$7 votos & 527 & 0,388 \\
            $\geq$8 votos & 354 & 0,381 \\
            $\geq$9 votos & 244 & 0,359 \\
            $\geq$10 votos & 164 & 0,323 \\
            $\geq$11 votos & 105 & 0,309 \\
            $\geq$12 votos & 64 & 0,293 \\
        \end{tabular}

        \includegraphics{kappa.png}
    \end{center}

    \note{El resultado obtenido para todos los tweets votados por más de una persona es de 0,6. También calculamos según una cantidad mínima de votos, y vemos que el valor de la medida kappa baja, como lo muestra la gráfica. Esto tiene sentido porque a más cantidad de votantes, más discrepancia posible hay.}

    \framebreak

    \begin{itemize}
        \item \large{0,612; acuerdo de nivel \textbf{medio-alto}}

        \begin{itemize}
            \item No hay unanimidad claramente
        \end{itemize}
    \end{itemize}

    \note{¿Cómo fue el acuerdo entre las personas entonces? Se considera que es bastante bueno, este número es considerado en general bastante bueno. Las personas en general están de acuerdo cuándo un tweet es humor. Aunque notar que es claro que en el humor no hay unanimidad en absoluto, lo cual era esperado.}
\end{frame}

\section{Clasificador}

\subsection{Metodología}
\begin{frame}
    \frametitle{Metodología}

    \begin{itemize}
        \item Se utilizan SVM, kNN, DT, GNB y MNB.
        \item Se divide en 80\% entrenamiento y 20\% evaluación.
        \item Se usa validación cruzada sobre el conjunto de entrenamiento para resultados intermedios durante el desarollo y el conjunto de evaluación para el resultado final.
    \end{itemize}
\end{frame}
\note{
    Se usa el conjunto de evaluación al final para no sesgarse demasiado a los resultados.
}

\subsection{Línea base}
\begin{frame}[allowframebreaks]
    \frametitle{Línea base}

    \begin{enumerate}
        \item BoW + MNB

        \item Clasificador que predice según lo que dice la mayoría (Negativo)
    \end{enumerate}

    \begin{center}
        \scriptsize
        \begin{tabular}{ c | r | r | r | r | r | r | r }
            & \multicolumn{1}{c |}{Precisión} & \multicolumn{1}{c |}{Recall} & \multicolumn{1}{c |}{$F_1$} & \multicolumn{1}{c |}{Prec. neg.} & \multicolumn{1}{c |}{Rec. neg.} & \multicolumn{1}{c |}{$F_1$ neg.} & \multicolumn{1}{c}{Acierto} \\
            \hline
            LB1 & 65,15 & 82,71 & 72,89 & 96,33 & 91,19 & 93,65 & 89,71 \\
            \hline
            LB2 & indefinido & 0,00 & indefinido & 82,49 & 100,00 & 90,41 & 82,49 \\
        \end{tabular}
    \end{center}
\end{frame}

\subsection{Características}

\begin{frame}
    \frametitle{Qué se busca}

    \begin{itemize}
        \item Contradicción y negatividad
        \item Formato e informalidad
        \item Orientación en personas
        \item Temas recurrentes en chistes
    \end{itemize}
\end{frame}

\begin{frame}
    \frametitle{Presencia de animales}

    \begin{itemize}
        \item Se conforma una lista de animales a partir de los chistes de animales de Chistes.com ($DIC_A$)
        \item Intersección de multiconjuntos:
    \end{itemize}

    \begin{center}
        \[
            PresenciaAnimales(tweet) = \frac{|tweet \cap DIC_A|}{\sqrt{|tweet|}}
        \]
    \end{center}
\end{frame}
\note{
    Características simples.
    $tweet$ es visto como una lista de tokens.
    En general todas las featuers se normalizan según el largo del tweet en tokens, para no sesgar.
}

\begin{frame}
    \frametitle{Jerga sexual}

    \begin{itemize}
        \item Se arma un diccionario de jerga sexual mediante \emph{Bootstrapping} en Twitter.
        \item Intersección de multiconjuntos:
    \end{itemize}

    \begin{center}
        \[
            JergaSexual(tweet) = \frac{|tweet \cap DIC_{JS}|}{\sqrt{|tweet|}}
        \]
    \end{center}
\end{frame}

\begin{frame}
    \frametitle{Primera persona}

    \begin{itemize}
        \item Se busca por palabras flexionadas en primera persona
    \end{itemize}
\end{frame}

\begin{frame}
    \frametitle{Segunda persona}

    \begin{itemize}
        \item Se busca por palabras flexionadas en segunda persona
    \end{itemize}
\end{frame}

\begin{frame}
    \frametitle{Distancia temática}

    \begin{itemize}
        \item Cercanía a un chiste de Chistes.com o cercanía a una oración de la Wikipedia.
        \item BoW + MNB
        \item Categorías
        \begin{itemize}
            \item Chistes cortos
            \item Adivinanzas
            \item Animales
            \item Atlantes
            \item Otros...
        \end{itemize}
    \end{itemize}
\end{frame}
\note{TODO: Aclarar qué es atlantes.}

\begin{frame}
    \frametitle{Diálogo}

    \begin{itemize}
        \item Si el tweet es un diálogo o no
    \end{itemize}
\end{frame}

\begin{frame}
    \frametitle{Preguntas-respuestas}

    \begin{itemize}
        \item Cantidad de preguntas seguidas de respuestas en el tweet.
    \end{itemize}
\end{frame}

\begin{frame}
    \frametitle{Palabras clave}

    \begin{itemize}
        \item Lista de palabras frecuentes en tweets
        \item Intersección de multiconjuntos:
    \end{itemize}

    \begin{center}
        \[
            PalabrasClave(tweet) = \frac{|tweet \cap DIC_{PF}|}{\sqrt{|tweet|}}
        \]
    \end{center}
\end{frame}

\begin{frame}
    \frametitle{Links}

    \begin{itemize}
        \item Cantidad de hipervínculos en el tweet
    \end{itemize}
\end{frame}

\begin{frame}
    \frametitle{Antónimos}

    \begin{itemize}
        \item Cantidad de pares de antónimos en el tweet:
    \end{itemize}

    \begin{center}
        \[
            Antonimos(tweet) = \frac{|\{pares de antonimos\}|}{\sqrt{|tweet|}}
        \]
    \end{center}
\end{frame}

\begin{frame}
    \frametitle{Hashtags}

    \begin{itemize}
        \item Cantidad de hashtags en el tweet
    \end{itemize}
\end{frame}

\begin{frame}
    \frametitle{Exclamación}

    \begin{itemize}
        \item Cantidad de signos de exclamación:
    \end{itemize}

    \begin{center}
        \[
            Exclamacion(tweet) = \frac{|\{signos de exclamacion\}|}{\sqrt{|tweet|}}
        \]
    \end{center}
\end{frame}

\begin{frame}
    \frametitle{Exclamación}

    \begin{itemize}
        \item Cantidad de palabras totalmente en mayúsculas:
    \end{itemize}

    \begin{center}
        \[
            PalabrasMayusculas(tweet) = \frac{|\{palabras\ mayusculas\}|}{\sqrt{|tweet|}}
        \]
    \end{center}
\end{frame}

\begin{frame}
    \frametitle{Negación}

    \begin{itemize}
        \item Cantidad de ``no'' en el tweet
    \end{itemize}
\end{frame}

\begin{frame}
    \frametitle{Palabras fuera del vocabluario (OOV)}

    \begin{itemize}
        \item Cantidad de palabras fuera del vocabulario, dividiento entre el total
        \item Son varias características:
        \begin{itemize}
            \item Freeling
            \item Freeling-Google
            \item Freeling-Wiktionary
            \item Wiktionary
        \end{itemize}
    \end{itemize}
\end{frame}
\note{
    Las distintas combinaciones de vocabulario debido a costo de uso y debido a ventajas y desventajas de cada uno. \textbf{Freeling} el más barato de usar, offline. Tiene un español más clasico, pero no tiene palabras ``nuevas''. \textbf{Wiktionary} tiene un término medio de todo. Es online, pero no limita su uso. Con \textbf{Google} se pueden obtener muchas palabras ``nuevas'' (como iPhone) y también detección de errores ortográficos, pero limita su uso.
}

\begin{frame}
    \frametitle{Palabras no españolas}

    \begin{itemize}
        \item Cantidad de palabras que tienen caracteres fuera del alfabeto español, normalizado según el total.
    \end{itemize}
\end{frame}

\subsection{Selección de características}
\begin{frame}[allowframebreaks]
    \frametitle{Selección de características}

    \begin{itemize}
        \item Se utiliza la Eliminación recursiva de atributos para seleccionar aquellos relevantes y no redundantes.
        \item Se descartan Negación, Palabras no españolas y Antónimos
    \end{itemize}

    \begin{center}
        \includegraphics{rfe.png}
    \end{center}
\end{frame}

\subsection{Resultados obtenidos}
\begin{frame}
    \frametitle{Resultados obtenidos}

    \begin{center}
        \scriptsize

        \begin{tabular}{ c | r | r | r | r | r | r | r }
            & \multicolumn{1}{c |}{Precisión} & \multicolumn{1}{c |}{Recall} & \multicolumn{1}{c |}{$F_1$} & \multicolumn{1}{c |}{Prec. neg.} & \multicolumn{1}{c |}{Rec. neg.} & \multicolumn{1}{c |}{$F_1$ neg.} & \multicolumn{1}{c}{Acierto} \\
            \hline
            LB1 & 61,68 & \textbf{84,63} & 71,35 & \textbf{96,59} & 89,23 & 71,35 & 88,45 \\
            \hline
            LB2 & indefinido & 0,00 & indefinido & 83,00 & \textbf{100,00} & 90,71 & 83,00 \\
            \hline
            SVM & \textbf{83,61} & 68,85 & \textbf{75,52} & 93,84 & 97,24 & \textbf{95,51} & \textbf{92,45} \\
            \hline
            DT & 66,51 & 67,54 & 67,02 & 93,33 & 93,03 & 93,18 & 88,85 \\
            \hline
            GNB & 57,49 & \textbf{78,17} & 66,25 & \textbf{95,17} & 88,16 & 91,53 & 86,46 \\
            \hline
            MNB & 84,76 & 60,02 & 70,27 & 92,27 & \textbf{97,79} & 94,95 & 91,37 \\
            \hline
            KNN & 81,26 & 66,31 & 73,03 & 93,35 & 96,87 & 95,08 & 91,67 \\
        \end{tabular}

        \vfill

        \begin{tabular}{ c | r | r }
            \textbf{son/clasif.} & Positivos & Negativos \\
            \hline
            Positivos & 842 & 381 \\
            \hline
            Negativos & 165 & 5805 \\
        \end{tabular}
    \end{center}
\end{frame}
\note{TODO: decir que es luego del escalado y demás}

\subsection{Otros análisis}

\subsubsection{Evaluación en el conjunto de entrenamiento}
\begin{frame}
    \frametitle{Evaluación en el conjunto de entrenamiento}

    \begin{itemize}
        \item Es una ``cota superior''
    \end{itemize}

    \begin{center}
        \scriptsize
        \begin{tabular}{ c | r | r | r | r | r | r | r }
            & \multicolumn{1}{c |}{Precisión} & \multicolumn{1}{c |}{Recall} & \multicolumn{1}{c |}{$F_1$} & \multicolumn{1}{c |}{Prec. neg.} & \multicolumn{1}{c |}{Rec. neg.} & \multicolumn{1}{c |}{$F_1$ neg.} & \multicolumn{1}{c}{Acierto} \\
            \hline
            SVM & 87,46 & 69,61 & 77,52 & 94,16 & 98,01 & 96,05 & 93,28 \\
            \hline
            DT & \textbf{99,96} & \textbf{98,82} & \textbf{99,38} & \textbf{99,76} & \textbf{99,99} & \textbf{99,88} & \textbf{99,80} \\
            \hline
            GNB & 58,06 & 77,65 & 66,44 & 95,21 & 88,79 & 91,89 & 86,94 \\
            \hline
            MNB & 84,56 & 58,93 & 69,46 & 92,26 & 97,85 & 94,97 & 91,67 \\
            \hline
            KNN & 86,98 & 71,47 & 78,47 & 94,49 & 97,86 & 96,15 & 93,47 \\
        \end{tabular}
    \end{center}
\end{frame}

\begin{frame}
    \frametitle{Evaluación en el conjunto de entrenamiento II}

    \begin{itemize}
        \item ¿Por qué DT no da 100\%? Debería darlo.
        \item Las características no discriminan completamente a la clase.
        \item ¿Hay errores en el corpus?
    \end{itemize}
\end{frame}

\begin{frame}[allowframebreaks]
    \frametitle{Inconsistencias en el corpus}

    \begin{itemize}
        \item Siguiendo la distancia mínima de edición en tweets (la granularidad es una palabra): se encontraron 367 pares de tweets ``parecidos'' pero con distinto valor de la clase objetivo.
    \end{itemize}

    \begin{example}
        Limpiar tu cuarto = 1\% limpieza. 30\% quejarse. 69\% jugar con lo que vas encontrando.
    \end{example}

    \begin{example}
        Limpiar tu cuarto:

        1\% limpieza.

        30\% quejarse.

        69\% jugar con lo que vas encontrando.
    \end{example}

    \note{
        En el primer caso son casi las mismas palabras, pero distinto espaciado y puntuación.
    }

    \framebreak

    \begin{itemize}
        \item Luego se buscan aquellos con mismos valores de atributos pero distinta clase: 30 pares encontrados.
    \end{itemize}

    \begin{example}
        \#TerminarUnaNotaDeSuicidioCon Soy Darks.
    \end{example}

    \begin{example}
        \#SiYoMeLlamaraKevinRoldan Me suicidaria.
    \end{example}
\end{frame}

\begin{frame}
    \frametitle{Inconsistencias en el corpus III}

    \begin{itemize}
        \item Hay inconsistencias en la anotación (era esperado)
        \item Hay una mejora sobre el corpus de entrenamiento quitando estas instancias:

        \begin{center}
            \scriptsize
            \begin{tabular}{ c | r | r | r | r | r | r | r }
                \textbf{SVM} & \multicolumn{1}{c |}{Precisión} & \multicolumn{1}{c |}{Recall} & \multicolumn{1}{c |}{$F_1$} & \multicolumn{1}{c |}{Prec. neg.} & \multicolumn{1}{c |}{Rec. neg.} & \multicolumn{1}{c |}{$F_1$ neg.} & \multicolumn{1}{c}{Acierto} \\
                \hline
                Antes & 87,46 & 69,61 & 77,52 & 94,16 & 98,01 & 96,05 & 93,28 \\
                \hline
                Después & 88,96 & 71,27 & 79,13 & 94,72 & 98,31 & 96,48 & 93,98 \\
            \end{tabular}
        \end{center}
    \end{itemize}

    \begin{itemize}
        \item Y en el de corpus de evaluación:

        \begin{center}
            \scriptsize
             \begin{tabular}{ c | r | r | r | r | r | r | r }
                \textbf{SVM} & \multicolumn{1}{c |}{Precisión} & \multicolumn{1}{c |}{Recall} & \multicolumn{1}{c |}{$F_1$} & \multicolumn{1}{c |}{Prec. neg.} & \multicolumn{1}{c |}{Rec. neg.} & \multicolumn{1}{c |}{$F_1$ neg.} & \multicolumn{1}{c}{Acierto} \\
                \hline
                Antes & 83,61 & 68,85 & 75,52 & 93,84 & 97,24 & 95,51 & 92,45 \\
                \hline
                Después & 85,71 & 69,21 & 76,58 & 94,20 & 97,75 & 95,94 & 93,08 \\
            \end{tabular}
        \end{center}
    \end{itemize}
\end{frame}

\subsubsection{Tweets censurados}
\begin{frame}
    \frametitle{Tweets censurados}

    \begin{itemize}
        \item El clasificador está sesgado a no conocer los tweets con contenido explícito
        \item Se anotan a mano los 303 tweets censurados y se agregan
        \item Hay una pequeña mejora:
        \begin{center}
            \scriptsize
            \begin{tabular}{ c | r | r | r | r | r | r | r }
                \textbf{SVM} & \multicolumn{1}{c |}{Precisión} & \multicolumn{1}{c |}{Recall} & \multicolumn{1}{c |}{$F_1$} & \multicolumn{1}{c |}{Prec. neg.} & \multicolumn{1}{c |}{Rec. neg.} & \multicolumn{1}{c |}{$F_1$ neg.} & \multicolumn{1}{c}{Acierto} \\
                \hline
                Antes & 83,61 & 68,85 & 75,52 & 93,84 & 97,24 & 95,51 & 92,45 \\
                \hline
                Después & 84,01 & 69,59 & 76,13 & 93,74 & 97,17 & 95,43 & 92,32 \\
            \end{tabular}
        \end{center}
        \item Un nuevo estudio de importancia de las características revela que no varía Jerga sexual: se agrega más variedad que jerga sexual.
    \end{itemize}
\end{frame}

\subsubsection{Restricción a cuentas humorísticas}
\begin{frame}
    \frametitle{Restricción a cuentas humorísticas}

    \begin{itemize}
        \item Es una tarea más difícil
        \item Igualmente se logran buenos resultados:

        \begin{center}
            \scriptsize
            \begin{tabular}{ c | r | r | r | r | r | r | r }
                & \multicolumn{1}{c |}{Precisión} & \multicolumn{1}{c |}{Recall} & \multicolumn{1}{c |}{$F_1$} & \multicolumn{1}{c |}{Prec. neg.} & \multicolumn{1}{c |}{Rec. neg.} & \multicolumn{1}{c |}{$F_1$ neg.} & \multicolumn{1}{c}{Acierto} \\
                \hline
                SVM & \textbf{81,87} & 73,83 & 77,64 & 78,47 & \textbf{85,36} & \textbf{81,77} & \textbf{79,92} \\
                \hline
                DT & 74,49 & 75,48 & 74,98 & 72,20 & 71,14 & 71,66 & 74,12 \\
                \hline
                GNB & 78,72 & 78,55 & \textbf{78,64} & 76,10 & 76,29 & 76,19 & 77,48 \\
                \hline
                MNB & 68,52 & \textbf{85,45} & 76,06 & \textbf{83,27} & 64,86 & 72,9 & 74,58 \\
                \hline
                KNN & 79,24 & 73,02 & 76,00 & 77,43 & 82,87 & 80,06 & 78,14 \\
            \end{tabular}
        \end{center}
    \end{itemize}
\end{frame}

\begin{frame}
    \frametitle{Métricas ponderadas según calificación}

    \begin{itemize}
        \item Tiene sentido sólo para los tweets que tuvieron votos y para la medida recall
        \item Promedio de estrellas, $PE_t = \frac{\sum_{i = 1}^{5} i \times v_{ti}}{v_t}$
        \item $recall_{ponderado} = \frac{\sum_{t \in VP} PE_t} {\sum_{t \in VP} PE_t + \sum_{t \in FN} PE_t} = 70.68\%$
        \item $\frac{recall_{ponderado}}{recall} = \frac{prom_{VP}}{prom_H} = 1.0266$
        \item Hay un (muy) leve sesgo del clasificador SVM a dar como positivos aquellos tweets que tienen mayor promedio de estrellas.
        \item Matriz de confusión:

        \begin{center}
            \begin{tabular}{ c | r | r }
                \textbf{son/clasificados} & Humor & No humor \\
                \hline
                Humor & 2,7227 & 2,4961 \\
                \hline
                No humor & 0,0256 & 0,0300 \\
            \end{tabular}
        \end{center}
    \end{itemize}
\end{frame}

\subsubsection{Clasificación según categorías de cuentas no humorísticas}
\begin{frame}
    \frametitle{Clasificación según categorías de cuentas no humorísticas}

    \begin{center}
        \scriptsize
        \begin{tabular}{ c | r | r | r | r | r | r | r }
            \textbf{SVM} & \multicolumn{1}{c |}{Precisión} & \multicolumn{1}{c |}{Recall} & \multicolumn{1}{c |}{$F_1$} & \multicolumn{1}{c |}{Prec. neg.} & \multicolumn{1}{c |}{Rec. neg.} & \multicolumn{1}{c |}{$F_1$ neg.} & \multicolumn{1}{c}{Acierto} \\
            \hline
            Noticias & \textbf{97,00} & \textbf{95,18} & \textbf{96,08} & \textbf{96,95} & \textbf{98,12} & \textbf{97,53} & \textbf{96,97} \\
            \hline
            Reflexiones & 94,95 & 82,99 & 88,57 & 83,95 & 95,27 & 89,25 & 88,92 \\
            \hline
            Curiosidades & 94,64 & 83,73 & 88,85 & 88,24 & 96,26 & 92,08 & 90,74 \\
        \end{tabular}
    \end{center}
\end{frame}

\subsubsection{Tweets dusosos}
\begin{frame}
    \frametitle{Tweets dudosos}

    \begin{center}
        \includegraphics{dudosos.png}
    \end{center}
\end{frame}

\section{Conclusiones}

\begin{frame}
    \frametitle{Conclusiones}
    
    \begin{itemize}[<+->]
    	\item[\checkmark] Se releva el Estado del arte
    	\item[\checkmark] Corpus anotado
    	\item[\checkmark] Clasificador con 83,61\% de precisión y 68,85\% de recall
    	\item[\checkmark] SVM fue el mejor
    	\item[\checkmark] Aprendizaje automático estadístico supervisado funcionó
    	\item[\checkmark] Diálogo característica más util, aunque en general todas aportan algo
    	\item[\checkmark] El clasificador da leves mejores resultados si el tweet es bien votado
    	\item[\checkmark] Noticias es lo que más se logra diferenciar respecto a humor
    	\item[\checkmark] Se encontraron inconsistencias en la anotación
    	\item[\checkmark] Las características no discriminan completamente al humor
    	\item[\checkmark] Es una tarea difícil
    \end{itemize}
\end{frame}

\subsection{Trabajo a futuro}
\begin{frame}
    \frametitle{Trabajo a futuro}
    Trabajo a futuro
\end{frame}


\begin{frame}
    \Huge{\centerline{Demo}}

    \begin{center}
        \href{http://demo.clasificahumor.com}{\beamergotobutton{demo.clasificahumor.com}}
    \end{center}
\end{frame}

\begin{frame}
    \Huge{\centerline{Questions?}}
\end{frame}

\end{document}
