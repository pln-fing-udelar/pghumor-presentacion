\section{State of the Art}

\subsection{Definitions}
\begin{frame}[allowframebreaks]
    \frametitle{Definitions}

    \begin{block}{Humor}
        A way to present, prosecute or comment the reality, highlighting the comic or ridiculous side.
    \end{block}

    \begin{block}{Joke}
        Funny saying or witticism.
    \end{block}

    \note{Primero, según el Diccionario la Real Academia Española, el humor es una manera de presentar la realidad, resaltando el lado cómico de las cosas. Recordemos que estre trabajo se trata sobre humor complementamente expresable dentro un texto. Es diferente a chiste, que es un dicho cómico, un cuento, aunque en este trabajo lo usamos de manera intercambiada con ``humor''.}

    \framebreak{}

    \begin{align*}
        \text{humor} \iff&  \text{comic} \\
        \text{irony} \iff& \text{opposition} \\
        \text{satire} \iff& \text{criticism + humor} \\
        \text{sarcasm} \iff& \text{derision} \\
        \text{ingenio}  \iff& \text{insight in humor}
    \end{align*}

    \note{Queremos diferenciarlo de otros conceptos similares, como ironía, que es significar lo contrario. Sátira, que es una crítica humorística generalmente en prosa. Sarcasmo, que es una forma de burla mordaz. Ingenio, que es darse cuenta de ciertos patrones. Nosotros queremos concentrarnos en lo que es la comicidad.}
\end{frame}

\subsection{Theories}
\begin{frame}
    \frametitle{Superiority Theory}

    \textbf{Superiority}: we always laugh at someone (lautgher = winning)
    \begin{example}
        --- How does Superman look like a honest politician? \\
        --- In which neither of the two exists.
    \end{example}
\end{frame}

\note{Viendo sobre lo que son las teorías del humor, empezamos con la de la superiodidad. La Teoría de la superidad afirma que siempre nos reímos de alguien, que hay alguien que es el objetivo del chiste, y pierde, y quien se ríe es el que gana.

En el ejemplo nos estaríamos ríendo de los políticos, al afirmar que no son honestos.}

\begin{frame}
    \frametitle{Relief Theory}

    To free yourself from subjects that generate tension, such as sex, death, etc.

    \begin{example}
        --- Welcome to McDonald’s, what do I give you? \\
        --- Shame! \\
        --- Mom! I'm working! \\
        --- Oops, excuse me ``mister licensed in graphic design'' \\
    \end{example}
\end{frame}

\note{Está también la Teoría del alivio, que afirma que existen temas que nos generan tensión, como el sexo, la muerte, conseguir un trabajo, casarse, etc., que la risa sirve para liberarlos.

Se muestra un ejemplo que ilustra este punto de vista.}

\begin{frame}
    \frametitle{Incongruity Resolution Theory}

    \textbf{Incongruity Resolution}: sudden perception of cognitive conflict, then finding the meaning
    \begin{example}
        --- Honey, we have been married for 30 years, we don't we get married? \\
        --- Better not, who will want to marry us?
    \end{example}
\end{frame}

\note{Por otro lado está la Teoría de la resolución de incongruencias, que afirma que cuando se llega al remate de un chiste se encuentra una inconcruencia, una incoherencia, pero luego somos capaces de resolverla, encontrando un sentido al texto, y así hay humor.

El ejemplo dice ``¿por qué no nos casamos?'' y ``¿quién se va a querer casar con nosotros?''. La ambigüedad hace ver una inconsistencia que luego podemos encontrarle un sentido. }

\begin{frame}
    \frametitle{Semantic Script Theory of Humor}

    Semantic scripts opposition.

    \begin{example}
        He broke her heart. She broke his iPhone. I think we all know who cried louder.
    \end{example}
\end{frame}

\note{Por último está la Teoría de los guiones semánticos del humor. Afirma que en el humor hay dos guiones que se ponen en conflicto entre sí.

En el ejemplo se habla de romper, de terminar un amor y a la vez de romper un dispositivo, comparándolos como si estuvieran en un mismo plano.

A continuación Matías va a mostrar los enfoques computacionales.
}

\subsection{Computational Approaches}
\begin{frame}
    \frametitle{Computational Approaches}

    Mainly based on feature engineering:

    \begin{itemize}
        \item Alliteration
        \item Ambiguity
        \item Antonymy
        \item Focused on people
        \item Sexual slang
        \item Negativity
        \item Keywords
        \item Perplexity --- OOV
    \end{itemize}
\end{frame}

\begin{frame}
    \frametitle{Alliteration}

    Noticeable repetition of phonemes:

    \begin{example}
        Of all the things I lost, I miss my mind the most
    \end{example}
\end{frame}

\begin{frame}
    \frametitle{Ambiguity}

    Of the words and of the sentences

    \begin{example}
        --- I saw a man-eating shark at the aquarium. \\
        --- That’s nothing. I saw a man eating herring at the deli.
    \end{example}
\end{frame}

\begin{frame}
    \frametitle{Antonymy}

    Opposition relationship between the meanings of two words

    \begin{example}
        --- What does Tarzan tell a mouse? \\
        --- So small an with a mustache! \\
        --- An what does the mouse tell Tarzan? \\
        --- So big and with a diaper! \\
    \end{example}
\end{frame}

\begin{frame}
    \frametitle{Focused on people}

    \begin{itemize}
        \item Constantly referring to events related to people.

        \item Words such as ``you'', ``I'', etc.
    \end{itemize}
\end{frame}

\begin{frame}
    \frametitle{Sexual slang}

    Humor based on sexual slang is popular.

    \begin{example}
        --- Hey, what's the meaning of ``hagamos el amor''? \\
        --- Let's make love. \\
        --- Ok, but then you tell me what it means.
    \end{example}
\end{frame}

\begin{frame}
    \frametitle{Negativity}

    Humor tends to be framed in negative scenarios.

    \begin{example}
        --- Dr., how do I do to live 100 years? \\
        --- No sex, no alcohol, no vices. \\
        --- And it works? \\
        --- I don't know, but I'm sure it will be long.
    \end{example}
\end{frame}

\begin{frame}
    \frametitle{Keywords}

    There are certain words that occur more often in jokes than in other texts.

    \begin{example}
        --- Mom, at school they call me Superman! \\
        --- Oh Jhonnie, you put your underpants on top of your pants again!
    \end{example}
\end{frame}

\begin{frame}
    \frametitle{Perplexity --- OOV}

    \begin{itemize}
        \item A language model is built based on narrations.

        \begin{itemize}
            \item Perplexity in jokes is higher.
        \end{itemize}

        \item Also, it's more common to find out of vocabulary words in jokes.
    \end{itemize}
\end{frame}

\begin{frame}
    \frametitle{Similar prior works}

    \begin{itemize}
        \item There are two similar works similar to this project:

        \begin{itemize}
            \item \emph{Making Computers Laugh: Investigations in Automatic Humor Recognition}, Mihalcea y Strapparava (2005)
            \item \emph{Recognizing Humor Without Recognizing Meaning}, Sjöbergh y Araki (2007)
        \end{itemize}

        \item Both in English and with one-liners.

        \item They extracted the jokes from Internet. An example:

        \begin{center}
            \emph{Beauty is in the eye of the beer holder.}
        \end{center}
    \end{itemize}
\end{frame}

\subsubsection{Main prior works}

\begin{frame}
    \frametitle{Main prior works}

    \begin{center}
        \scriptsize
        \begin{tabular}{>{\centering\arraybackslash}m{1.9cm} m{4cm} m{4cm}}
            & \multicolumn{1}{c}{Mihalcea y Strapparava (2005)} & \multicolumn{1}{c}{Sjöbergh y Araki (2007)} \\
            \midrule
            Positive samples & 16.000 jokes & 6.100 jokes \\
            \midrule
            Negative samples & Sentences from BNC, news headlines and proverbs & 6.000 different sentences from BNC \\
            \midrule
            Accuracy & 96,95\% with news headlines, 79,15\% with sentences from BNC y 84,82\% with proverbs & 85,40\% \\
            \midrule
            Features & Alliteration, Antonymy and Sexual slang & Alliteration, Ambiguity, Antonymy, Words focused on people and Keywords
        \end{tabular}

        \vfill

        Results are not comparable.
    \end{center}
\end{frame}
