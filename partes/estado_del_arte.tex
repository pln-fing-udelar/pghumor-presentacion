\section{Estado del arte} % A subsection can be created just before a set of slides with a common theme to further break down your presentation into chunks

\subsection{Definiciones}
\begin{frame}[allowframebreaks]
	\frametitle{Definiciones}

	\begin{block}{Humor $\not\subset$ Gracioso}
	Modo de presentar, enjuiciar o comentar la realidad, resaltando el lado cómico, risueño o ridículo de las cosas.  
	\end{block}
	\begin{block}{Chiste $\subset$ Humor}
	Dicho u ocurrencia graciosa.
	\end{block}

	\framebreak

	\begin{block}{Ironía}
		Figura retórica que consiste en dar a entender lo contrario a lo que se dice.
	\end{block}
	\begin{block}{Sátira}
		Composición en verso o prosa cuyo objetivo es censurar o ridiculizar a alguien o algo
	\end{block}
	\begin{block}{Sarcasmo}
		Burla sangrienta, ironía mordaz y cruel con que se ofende o maltrata a alguien o algo.
	\end{block}
	\begin{block}{Ingenio}
		Chispa, talento para ver y mostrar rápidamente el aspecto gracioso de las cosas.
	\end{block}

	\framebreak
		
	\begin{eqnarray*} %% Do avoid eqnarray if possible.
    	\text{humor} \iff&  \text{cómico}\\
	    \text{ironía} \iff& \text{significar lo contrario}\\
	    \text{sátira} \iff& \text{crítica + humor}\\
	    \text{sarcasmo} \iff& \text{burla}\\
	    \text{ingenio}  \iff& \text{perspicacia en el humor}
	\end{eqnarray*}
\end{frame}

\subsection{Teorías}
\begin{frame}
	\frametitle{Teoría de la superioridad}

	\textbf{Superioridad}: siempre nos reímos de alguien. (risa = ganar)
	\begin{example}
		— ¿En qué se parece Superman a un político honesto?\\
		— En que ninguno de los dos existe.
	\end{example}
\end{frame}

\begin{frame}
\frametitle{Teoría del alivio}
	\textbf{Aliviarse}: de temas que generan tensión, como el sexo, la muerte, etc.
	\begin{example}
		—Bienvenida a McDonald’s, ¿qué le doy?\\
		—¡Vergüenza!\\
		—¡Mamá! ¡Estoy trabajando!\\
		—Uy, perdóneme ``señor licenciado en diseño gráfico''\\
	\end{example}
\end{frame}

\begin{frame}
\frametitle{Teoría de la resolución de incongruencias}
	\textbf{Resolución de incongruencia}: percepción repentina de conflicto cognitivo, encontrando luego un sentido
	\begin{example}
		— Mi amor llevamos 30 años juntos, ¿por qué no nos casamos?\\
		— Mejor no, ¿quién se va a querer casar con nosotros?
	\end{example}
\end{frame}

\begin{frame}
\frametitle{Teoría de la violación}
	\textbf{Violación}: de cómo pienso que van a ser las cosas, aunque la situación parece normal
	\begin{example}
		Mi iPhone 6 Plus lo cuido tanto que todavía está en su caja. . . en la tienda. . . en el shopping. . .
	\end{example}
\end{frame}

\begin{frame}
\frametitle{Teorías sociológicas}
	\begin{itemize}
		\item Fines sociológicos de mantenimiento
		\item Fines sociológicos de negociación
		\item Fines sociológicos de cambio de contexto
	\end{itemize}
\end{frame}

\begin{frame}
\frametitle{Teoría de los guiones semánticos del humor}
	Oposición de guiones semánticos
	\begin{example}
		Él rompió su corazón. Ella rompió su PlayStation 4. Creo que todos sabemos quién lloró más fuerte.
	\end{example}
\end{frame}

\begin{frame}
\frametitle{Teoría general del humor verbal}
	Respuesta innovadora y familiar a un estímulo
	\begin{example}
		— Cariño, dame el bebé.\\
		— Espera a que llore.\\
		— ¿A que llore?. ¿Por qué?\\
		— ¡Porque no lo encuentro!\\
	\end{example}
\end{frame}

\subsection{Enfoques computacionales}
\begin{frame}
\frametitle{Paragraphs of Text}
Enfoques computacionales
\end{frame}
