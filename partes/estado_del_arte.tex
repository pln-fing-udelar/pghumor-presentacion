\section{Estado del arte}

\subsection{Definiciones}
\begin{frame}[allowframebreaks]
    \frametitle{Definiciones}

    \begin{block}{Humor}
        Modo de presentar, enjuiciar o comentar la realidad, resaltando el lado cómico, risueño o ridículo de las cosas.  
    \end{block}
    \begin{block}{Chiste}
        Dicho u ocurrencia graciosa.
    \end{block}

    \framebreak
        
    \begin{align*}
        \text{humor} \iff&  \text{cómico} \\
        \text{ironía} \iff& \text{significar lo contrario} \\
        \text{sátira} \iff& \text{crítica + humor} \\
        \text{sarcasmo} \iff& \text{burla} \\
        \text{ingenio}  \iff& \text{perspicacia en el humor}
    \end{align*}
\end{frame}

\subsection{Teorías}
\begin{frame}
    \frametitle{Teoría de la superioridad}

    \textbf{Superioridad}: siempre nos reímos de alguien (risa = ganar)
    \begin{example}
        --- ¿En qué se parece Superman a un político honesto? \\
        --- En que ninguno de los dos existe.
    \end{example}
\end{frame}

\begin{frame}
    \frametitle{Teoría del alivio}

    \textbf{Aliviarse}: de temas que generan tensión, como el sexo, la muerte, etc.
    \begin{example}
        --- Bienvenida a McDonald’s, ¿qué le doy? \\
        --- ¡Vergüenza! \\
        --- ¡Mamá! ¡Estoy trabajando! \\
        --- Uy, perdóneme ``señor licenciado en diseño gráfico'' \\
    \end{example}
\end{frame}

\begin{frame}
    \frametitle{Teoría de la resolución de incongruencias}

    \textbf{Resolución de incongruencia}: percepción repentina de conflicto cognitivo, encontrando luego un sentido
    \begin{example}
        --- Mi amor llevamos 30 años juntos, ¿por qué no nos casamos? \\
        --- Mejor no, ¿quién se va a querer casar con nosotros?
    \end{example}
\end{frame}

\begin{frame}
    \frametitle{Teoría de los guiones semánticos del humor}

    Oposición de guiones semánticos

    \begin{example}
        Él rompió su corazón. Ella rompió su PlayStation 4. Creo que todos sabemos quién lloró más fuerte.
    \end{example}
\end{frame}

\subsection{Enfoques computacionales}
\begin{frame}
    \frametitle{Enfoques computacionales}

    \begin{itemize}
        \item Basados en características

        \item Característica: propiedad medible de un texto

        \item Existen trabajos anteriores que estudian diferentes características.
    \end{itemize}
\end{frame}

\subsubsection{Características}

\begin{frame}
    \frametitle{Características}

    \begin{itemize}
        \item Aliteración
        \item Ambigüedad
        \item Antonimia
        \item Centrado en personas
        \item Jerga sexual
        \item Negatividad
        \item Palabras clave
        \item Perplejidad --- OOV
    \end{itemize}
\end{frame}

\begin{frame}
    \frametitle{Aliteración}

    Repetición notoria de fonemas

    \begin{example}
        Cuando pasé por tu puerta me tiraste una flor, la próxima vez que pase, ¡sin maceta por favor!
    \end{example}
\end{frame}

\begin{frame}
    \frametitle{Ambigüedad}
    
    De las palabras y de la oración

    \begin{example}
        La perra de mi vecina me ladró.
    \end{example}
\end{frame}

\begin{frame}
    \frametitle{Antonimia}
    
    Relación de oposición entre los significados de dos palabras (útil --- inútil)

    \begin{example}
        --- ¿Qué le dice Tarzán a un ratón? \\
        --- ¡Tan pequeño y con bigote! \\
        --- ¿Y qué le dice el ratón a Tarzán? \\
        --- ¡Tan grandote y con pañal! \\
    \end{example}
\end{frame}

\begin{frame}
    \frametitle{Centrado en personas}
    
    \begin{itemize}
        \item Constantemente referenciando a escenarios relacionados con personas.
        
        \item Palabras como “tú”, “yo”, etc.
    \end{itemize}
\end{frame}

\begin{frame}
    \frametitle{Jerga sexual}
    
    El humor basado en jerga sexual es altamente popular

    \begin{example}
        --- Che, ¿qué significa “let’s fuck”? \\
        --- Tengamos sexo. \\
        --- Bueno, pero después me decís qué significa.
    \end{example}
\end{frame}

\begin{frame}
    \frametitle{Negatividad}
    
    El humor tiene una tendencia hacia las connotaciones negativas

    \begin{example}
        --- Dr., ¿cómo hago para vivir 100 años? \\
        --- Nada de sexo, alcohol ni vicios. \\
        --- ¿Y funciona? \\
        --- No sé, pero seguro que se le va a hacer largo.
    \end{example}
\end{frame}

\begin{frame}
    \frametitle{Palabras clave}
    
    Hay palabras que son más mencionadas en textos humorísticos que en textos de no humorísticos, y viceversa.
    
    \begin{example}
        --- Mamá, ¡en la escuela me dicen Superman! \\
        --- Ay Jaimito, ¡otra vez te pusiste los calzoncillos encima de los pantalones!
    \end{example}
\end{frame}

\begin{frame}
    \frametitle{Perplejidad --- OOV}
    
    \begin{itemize}
        \item Se construye un modelo del lenguaje a partir de narraciones

        \begin{itemize}
            \item La perplejidad en textos de humor es mayor
        \end{itemize}

        \item Es más frecuente encontrar palabras fuera del vocabulario (OOV) en textos de humor
    \end{itemize}
\end{frame}

\begin{frame}
    \frametitle{Trabajos similares}

    \begin{itemize}
        \item Existen dos trabajos similares a este proyecto:

        \begin{itemize}
            \item \emph{Making Computers Laugh: Investigations in Automatic Humor Recognition}, Mihalcea y Strapparava (2005)
            \item \emph{Recognizing Humor Without Recognizing Meaning}, Sjöbergh y Araki (2007)
        \end{itemize}
        \item Textos en inglés y de una sola oración.
    \end{itemize}
\end{frame}

\subsubsection{Trabajos anteriores principales}

\begin{frame}
    \frametitle{Trabajos anteriores principales}
    
    \begin{center}
        \scriptsize
        \begin{tabular}{>{\centering\arraybackslash}m{1.9cm} | m{4cm} | m{4cm}}
            & \multicolumn{1}{c |}{Mihalcea y Strapparava (2005)} & \multicolumn{1}{c}{Sjöbergh y Araki (2007)} \\
            \hline
            Ejemplos negativos & Oraciones del BNC, titulares de noticias y proverbios & Oraciones distintas del BNC \\
            \hline
            Acierto & 96,95\% con titulares de noticias, 79,15\% con oraciones del BNC y 84,82\% con los proverbios & 85,40\% \\
            \hline
            Características & Aliteración, Antonimia y Jerga sexual & Aliteración, Ambigüedad, Antonimia y Palabras centradas en las personas, Palabras clave
        \end{tabular}

        \vfill

        Los resultados no son comparables.
    \end{center}
\end{frame}
