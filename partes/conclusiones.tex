\section{Conclusiones}

\begin{frame}
    \frametitle{Conclusiones}
    
    \begin{itemize}[<+->]
    	\item[\checkmark] Se releva el Estado del arte
    	\item[\checkmark] Corpus anotado
    	\item[\checkmark] Clasificador con 83,61\% de precisión y 68,85\% de recall
    	\item[\checkmark] SVM fue el mejor
    	\item[\checkmark] Aprendizaje automático estadístico supervisado funcionó
    	\item[\checkmark] Diálogo característica más util, aunque en general todas aportan algo
    	\item[\checkmark] El clasificador da leves mejores resultados si el tweet es bien votado
    	\item[\checkmark] Noticias es lo que más se logra diferenciar respecto a humor
    	\item[\checkmark] Se encontraron inconsistencias en la anotación
    	\item[\checkmark] Las características no discriminan completamente al humor
    	\item[\checkmark] Es una tarea difícil
    \end{itemize}
\end{frame}

\subsection{Trabajo a futuro}
\begin{frame}
    \frametitle{Trabajo a futuro}
    
    \begin{itemize}[<+->]
    	\item Características más complejas
    	\item Características independientes
    	\item Regresión en el promedio de votación
    	\item Estudiar el humor y la anotación según estratos sociales
    \end{itemize}
\end{frame}
