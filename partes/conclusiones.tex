\section{Conclusiones}

\begin{frame}
    \frametitle{Conclusiones}
    
    \begin{itemize}
    	\item[\checkmark] Corpus anotado
    	\item[\checkmark] Clasificador con 83,61\% de precisión y 68,85\% de recall
    	\item[\checkmark] SVM fue el mejor
    	\item[\checkmark] Diálogo característica más util, aunque en general todas aportan algo
    	\item[\checkmark] Hay una leve mejora si el tweet tiene promedio de estrellas alto
    	\item[\checkmark] Noticias es lo que más se logra diferenciar respecto a humor
    	\item[\checkmark] Se encontraron inconsistencias en la anotación
        \item[\checkmark] Características de formato y temática llevan a buenos resultados
    	\item[\checkmark] Las características no discriminan completamente al humor
    	\item[\checkmark] Es una tarea difícil
    \end{itemize}
\end{frame}

\subsection{Trabajo a futuro}
\begin{frame}
    \frametitle{Trabajo a futuro}
    
    \begin{itemize}
        \item Más características
    	\item Características más complejas
    	\item Características independientes
    	\item Regresión en el promedio de estrellas
    	\item Estudiar el humor y la anotación según estratos sociales
        \item Generar humor
    \end{itemize}
\end{frame}
