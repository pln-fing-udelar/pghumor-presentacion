\section{Construcción del corpus}

\subsection{Extracción}
\begin{frame}
    \frametitle{Extracción}

    \begin{itemize}
        \item Humorístico: se busca en Twitter por la palabra clave \emph{chistes} y se eligen cuentas, llegando a 16.488 tweets.
        \item No humorístico: cuentas de noticias, frases filosóficas y curiosidades, alcanzando los 22.875 tweets.
    \end{itemize}
\end{frame}

\subsection{Anotación}
\begin{frame}[allowframebreaks]
    \frametitle{Anotación}

    \begin{itemize}
        \item Naturalmente se etiquetarían los tweets según su tipo de cuenta; pero se encuentran inconsistencias.

        \begin{itemize}
            \item Hay que anotarlos a mano.
        \end{itemize}

        \item Excesiva cantidad de tweets para anotar.

        \begin{itemize}
            \item Se crea una aplicación para que usuarios los anoten.
        \end{itemize}
    \end{itemize}

    \vspace{1cm}

    \begin{center}
        \bf
        Los usuarios definen al humor.
    \end{center}

    \framebreak

    \begin{itemize}
        \item Cantidad de clases a considerar
        \item Contenido explícito
        \item Eficiencia
        \item Algoritmo de selección
    \end{itemize}

    \framebreak

    \begin{center}
        \begin{columns}[c]
            \begin{column}[c]{0.45\textwidth}
                \centering
                \includegraphics[frame, height=3.5cm]{pagina.png}
            \end{column}

            \begin{column}[c]{0.45\textwidth}
                \centering
                \includegraphics[frame, height=7cm]{app.png}
            \end{column}
        \end{columns}
    \end{center}
\end{frame}

\subsubsection{Resultado de la anotación}
\begin{frame}[allowframebreaks]
    \frametitle{Resultado de la anotación}

    \begin{itemize}
        \item[+] 60k votaciones recibidas
        \item[--] 20k votaciones eliminadas
        \item[--] 6,5k votaciones “ver otro” (\emph{“skip”})
        \item[=] 33,5k votos considerados
    \end{itemize}

    \framebreak

    \begin{center}
        \includegraphics{votos_por_calificacion_torta.png}

        \includegraphics{histograma.png}
    \end{center}
\end{frame}

\subsubsection{Humor según la votación}

\begin{frame}[allowframebreaks]
    \frametitle{Humor según la votación}

    Porcentaje de votos positivos: $\frac{\#\{votos\ positivos\}}{\#\{total\ votos\}}$

    \begin{center}
        \includegraphics[height=3.75cm]{histograma_porcentaje_humor.png}
    \end{center}

    \begin{itemize}
        \item $[60\%, 100\%] \Rightarrow$ \textbf{Humor}
        \item $(30\%, 60\%) \Rightarrow$ \textbf{Dudoso}
        \item $[0\%, 30\%] \Rightarrow$ \textbf{No humor}
    \end{itemize}

    \framebreak

    \begin{center}
        \includegraphics[height=6.5cm]{grupos.png}
    \end{center}
\end{frame}

\subsubsection{Concordancia entre los anotadores}

\begin{frame}[allowframebreaks]
    \frametitle{Concordancia entre los anotadores}

    \begin{itemize}
        \item Se quiere saber qué tan de acuerdo estuvieron las personas a la hora de votar.
        \item Se utiliza la medida kappa de Fleiss.
        \item kappa evalúa cuán mejor es la votación respecto a una al azar, siendo lo mejor posible 1 y siendo 0 una votación al azar.
    \end{itemize}

    \framebreak

    \begin{center}
        \begin{tabular}{ c | r | c }
            tweets considerados & \#tweets & $\kappa$ \\
            \hline
            $\geq$2 votos & 8.320 & 0,612 \\
            $\geq$3 votos & 4.309 & 0,523 \\
            $\geq$4 votos & 2.273 & 0,469 \\
            $\geq$5 votos & 1.331 & 0,434 \\
            $\geq$6 votos & 805 & 0,406 \\
            $\geq$7 votos & 527 & 0,388 \\
            $\geq$8 votos & 354 & 0,381 \\
            $\geq$9 votos & 244 & 0,359 \\
            $\geq$10 votos & 164 & 0,323 \\
            $\geq$11 votos & 105 & 0,309 \\
            $\geq$12 votos & 64 & 0,293 \\
        \end{tabular}

        \includegraphics{kappa.png}
    \end{center}

    \framebreak

    \begin{itemize}
        \item \large{0,612; acuerdo de nivel \textbf{medio-alto}}

        \begin{itemize}
            \item No hay unanimidad claramente
        \end{itemize}
    \end{itemize}
\end{frame}
