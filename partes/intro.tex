\section{Introducción} 

\subsection{Motivación}

\begin{frame}[allowframebreaks]
    \frametitle{Motivación}
    \begin{itemize}
        \item La risa caracteriza al ser humano como especie.
        \item Componente esencial en la comunicación humana.
        \item Humor como pieza fundamental en la interacción persona-computadora.
    \end{itemize}

    \framebreak
    
    \begin{itemize}
        \item El humor ha sido estudiado desde el punto de vista psicológico, cognitivo y lingüístico.
        \item ¿Pero desde el punto de vista computaciónal?
    \end{itemize}
    Algunos trabajos previos existen, pero se está aún lejos de concretar una caracterización del humor que permita su reconocimiento y generación automática.
\end{frame}

\begin{frame}
    \frametitle{Dificultad}
    
    --- ¿Tenés WiFi? \\
    --- Claro. \\
    --- ¿Cuál es la clave? \\
    --- Tener dinero y pagarlo. \\
\end{frame}

\subsection{Objetivos}

\begin{frame}
    \frametitle{Objetivos}
    \begin{itemize}
        \item Construir un clasificador de humor en textos en español utilizando métodos de aprendizaje automático
            \begin{itemize}
                \item En particular en tweets
            \end{itemize}
        \item Construir un corpus de tweets en español
    \end{itemize}
\end{frame}
