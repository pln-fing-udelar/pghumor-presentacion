\section{Introducción}

\subsection{Motivación}

\begin{frame}[allowframebreaks]
    \frametitle{Motivación}

    \begin{itemize}
        \item La risa caracteriza al ser humano como especie.

        \item Componente esencial en la comunicación humana.

        \item Humor como pieza fundamental en la interacción persona-computadora.
    \end{itemize}

    \note{La risa caracteriza al ser humano como especia, y el humor potencialmente la origina.

    El humor es a su vez un componente esencial en la comunicación humana. Permite que no seamos infelices a la vez que genera un ambiente más distendido.

    Es una pieza fundamental en nuestra interacción con la computadora. Puede generar interfaces más amigables y más adaptadas.}

    \framebreak{}

    \begin{itemize}
        \item El humor ha sido estudiado desde el punto de vista psicológico, cognitivo y lingüístico.

        \item ¿Pero desde el punto de vista computaciónal?

        \item Paso intermedio para la generación.
    \end{itemize}

    \note{El humor ha sido ampliamente estudiado desde el punto de vista psicológico, sociológico, cognitivo y lingüístico, ¿pero desde el punto de vista computacional? Trabajos previos existen, pero son pocos y con una versión reducida del problema.

    Adicionalmente nos motiva que este trabajo puede ser utilizado como paso intermedio en la generación de humor o generación de chistes. Es decir, usar el reconocimiento para luego intentar generarlo.}
\end{frame}

\subsection{Dificultad}

\begin{frame}
    \frametitle{Dificultad}

    --- ¿Tenés WiFi? \\
    --- Claro. \\
    --- ¿Cuál es la clave? \\
    --- Tener dinero y pagarlo. \\
\end{frame}

\note{Para mostrar la dificultad del problema se trae al frente el siguiente ejemplo. Para que una computadora logre interpretar este texto es necesario que conozca cada uno de los conceptos incluidos, debe tener conocimiento y contexto sobre el mundo. Debe saber por ejemplo qué es WiFi, que lleva contraseña, que contraseña es lo mismo que clave (que son sinónimos). Que la persona pregunta por la contraseña, y que la otra persona responde cómo conseguir WiFi en lugar de la respuesta a la pregunta original. Que por lo tanto la segunda pregunta es ambigua. En definitiva darse cuenta que esto no es serio: una respuesta correcta a preguntar por una contraseña no es tener dinero y pagarlo.}

\note{En el presente trabajo no se pretende interpretar el idioma español, que es algo muy difícil. Se pretende detectar humor a través de \textbf{características}, accidentales, pero que juntas pueden dar buenos resultados. Por ejemplo, se observa que hay \textbf{ambigüedad} en este chiste, hay dos interpretaciones posibles. Esto es una característica interesante a mirar. Por otro lado hay una \textbf{oposición de temas}: se comienza hablando del WiFi y se termina hablado de pagar algo. También observar que este tweet es un \textbf{diálogo}: ¿quién tweetearía algo que es un diálogo ambiguo y que opone temas? Debe ser humorístico. Las características que se miran pueden ser simples, pero combinadas tienen un gran poder.}

\subsection{Humor}

\begin{frame}
    \frametitle{Humor}

    \begin{itemize}
        \item ¿Qué es el humor?
        \item No es algo objetivo ni fácil de definir
    \end{itemize}
\end{frame}

\note{El humor; ¿qué es el humor? Es algo muy subjetivo. Depende a quién le preguntes es la respuesta que vas a obtener? Además no es fácil de definir por las personas.

En el presente trabajo nos concentramos en el humor que pueda ser completamente expresado de forma escrita.}

\subsection{Tipo de textos}

\begin{frame}
    \frametitle{Tipo de textos}

    \begin{itemize}
        \item En un texto largo hay que determinar el alcance
        \item[$\Rightarrow$] Los \emph{tweets} son cortos y fácilmente conseguidos

        \begin{itemize}
            \item Usualmente con errores gramaticales
        \end{itemize}
    \end{itemize}
\end{frame}

\note{Un punto a tener en cuenta es qué tipos de textos vamos a tratar. Podríamos por ejemplo tratar un artículo, sin embargo podría ocurrir que hay humor en una parte sí, en otra no, sigue un pedazo con humor distinto, etc. Esto quiere decir que nos metemos con la tarea de determinar el \textbf{alcance} del humor, duplicando la complejidad. Para no meternos con este contratiempo decidimos tratar textos de tamaño corto y decidimos tratar \textbf{tweets}. Ellos son publicaciones de la red social Twitter que la gente hace al rededor del mundo. Cumplen que además son fácilmente conseguidos. Por otro lado, como su uso es en su mayoría informal, en general contienen errores gramaticales, dificultando el uso de herramientas con ellos.}

\subsection{Objetivos}

\begin{frame}
    \frametitle{Objetivos}

    \begin{itemize}
        \item Construir un clasificador de humor en tweets en español utilizando métodos de aprendizaje automático

        \begin{itemize}
            \item Se precisa un corpus de tweets en español
        \end{itemize}
    \end{itemize}
\end{frame}

\note{Concretando, el objetivo de este trabajo es construir un clasificador de humor en tweets en idioma español utilizando métodos de aprendizaje automático, área que luego vamos a detallar. Para esto es necesario primero contar con un corpus, un conjunto de textos, de tweets en español marcados como humorísticos o como no humorísticos.}
